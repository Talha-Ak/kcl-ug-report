\chapter{Requirements}

The aim of this project is to implement a compiler for a small functional language, targeting the
LLVM IR. The language syntax is largely inspired by the language presented in the KCL module
6CCS3CFL (which in turn was inspired by Scala), with additional features.

The functional language should be able to express and support the following concepts:

\begin{itemize}
    \onehalfspacing
    \item \textbf{Arithmetic} and \textbf{Boolean} expressions
    \item \textbf{Variable Bindings}
    \item \textbf{If} expressions
    \item \textbf{Printing} to the console
    \item \textbf{Functions} and \textbf{Function Application}
    \item \textbf{Higher-Order Functions} (i.e. functions that take other functions as arguments)
    \item Basic atomic \textbf{Types} (e.g. \mintinline{scala}{Int}, \mintinline{scala}{Bool},
          \dots)
    \item A basic form of \textbf{Pattern Matching} on \textbf{Algebraic Data Types}
\end{itemize}

The compiler should be able to take a program written in this language and produce the semantically
equivalent LLVM IR. The conversion should be performed via the following steps:

\begin{enumerate}
    \onehalfspacing
    \item Parsing the source program into an Abstract Syntax Tree (AST)
    \item Transforming the AST into an appropriate intermediate representation (IR)
    \item Applying relevant transformations (e.g.,\ closure conversion, hoisting, \dots) to
          the IR
    \item Converting the intermediate representation into LLVM IR
\end{enumerate}

