\chapter{Requirements}
\label{ch:requirements}

The aim of this project is to implement a compiler for a small functional language, targeting the
LLVM IR. The language syntax is largely inspired by the language presented in the KCL module
6CCS3CFL (which in turn was inspired by Scala), with additional features.

The functional language should be able to express and support the following concepts (with
examples):

\begin{itemize}
    \item \textbf{Arithmetic} and \textbf{Boolean} expressions \\
          (Operator precedence and associativity must be respected.)
          \begin{code}{scala}
              1 + 2 * 3 < 4 && 5 >= 6
          \end{code}

    \item \textbf{Variable Bindings}
          \begin{code}{scala}
              val x = 1; val y = 2; val z = x + y
          \end{code}

    \item \textbf{If} expressions with optional else clauses \\
          (Care must be taken to avoid the dangling else problem.)
          \begin{code}{scala}
              if (x < 0) 100 else if (x > 0) -100 else 50
          \end{code}

    \item \textbf{Printing} to the console
          \begin{code}{scala}
              print(9 * (5 + 3))
          \end{code}

    \item \textbf{Functions} and \textbf{Function Application}
          \begin{code}{scala}
              def foo(x: Int, y: Int): Int = x + y;
              foo(1, 2)
          \end{code}

    \item \textbf{Higher-Order Functions} (i.e. functions that take other functions as arguments)
          \begin{code}{scala}
              def apply(f: Int => Int, x: Int): Int = f(x);
              def increment(x: Int): Int = x + 1;
              apply(increment, 1)
          \end{code}

    \item \textbf{Closures} (i.e. functions that capture variables from their surrounding scope)
          \begin{code}{scala}
                def makeIncrementer(x: Int): Int => Int = {
                    def incrementer(y) = x + y;
                    incrementer
                }
                val incrementBy5 = makeIncrementer(5);
                incrementBy5(10)
          \end{code}

    \item Basic atomic \textbf{Types}
          (e.g. \mintinline{scala}{Int}, \mintinline{scala}{Bool}, \mintinline{scala}{Float}, \dots)
          \begin{code}{scala}
              val x: Int = 1; val y: Bool = true
          \end{code}

    \item User-defined \textbf{Structure Types}
          \begin{code}{scala}
              type Point = { x: Int, y: Int };
              val p: Point = { x = 1, y = 2 }
          \end{code}

    \item A basic form of \textbf{Pattern Matching} on \textbf{Enumerated Types}
          \begin{code}{scala}
              enum Priority = Low | Medium | High;

              def foo(p: Priority): Int = p match {
                  case Low => 1
                  case Medium => 2
                  case High => 3
              }
          \end{code}
\end{itemize}

The compiler should be able to take a program written in this language and produce the semantically
equivalent LLVM IR.

The emitted LLVM IR should be compiled and executed using the LLVM toolchain. If possible, this
should result in the final compiler being able to compile to multiple target architectures, such as
x86 and ARM. Access to the LLVM toolchain should also provide the ability to apply various
optimisations to the generated LLVM IR without the need to implement them manually.
