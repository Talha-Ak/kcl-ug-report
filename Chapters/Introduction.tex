\chapter{Introduction}

Functional languages are a class of programming languages that stem from the lambda calculus,
treating computation as the evaluation of functions, avoiding side effects and mutable state.
Benefits of this paradigm include the ability to apply formal mathematical reasoning about programs,
while encouraging modularity and code reuse. Functional languages are often easier to reason about,
as the lack of side effects makes programs more predictable and easier to test. This predictability
also makes functional languages more amenable to automatic parallelisation, as the lack of side
effects means that functions can be executed in any order without affecting the result.

Advantages of this paradigm have even led to the adoption of some functional language features in
other popular languages, including Java, C\#, and Rust. During the development of this project, new
languages have been released that take heavy inspiration from the functional paradigm, such as the
Erlang runtime-based language Gleam.

To facilitate programming in this paradigm, modern functional languages typically provide a number
of features, such as higher-order functions, pattern matching, and algebraic data types
\autocite{hudak1989conception}. Typical imperative constructs such as loops and mutable variables
are also discouraged, instead favouring recursion and immutable data structures. These
characteristics make compilers for functional languages an interesting case study, as more work is
offloaded to the compiler for transforming programs into efficient imperative machine code -- code
that is in a different paradigm to that of a functional language.

One method of compiling functional languages is via the LLVM toolchain, which provides a collection
of utilities for compiling, optimising, and linking programs. For language designers, one can create
a frontend that compiles a language into the LLVM Intermediate Representation (IR), which can then
be compiled into machine code for a variety of architectures by leveraging the LLVM backend. This
approach can be preferable in comparison to:

\begin{itemize}
    \item \textbf{Compiling to native machine code:} While theoretically the most efficient method
          of compiling a program, this approach requires the language designer to implement a
          backend for each target architecture. This can be time-consuming and error-prone, as the
          language designer must be familiar with the intricacies of each architecture.

    \item \textbf{Compiling to a virtual machine (VM):} This approach involves compiling the program
          to a VM runtime, which can then be executed on any machine that has the VM installed.
          While this approach is portable, it can be less performant than compiling to native
          machine code, as the program must be interpreted by the VM at runtime. Additionally, the
          language designer must either implement a VM runtime for each target architecture
          (providing similar challenges to compiling to native machine code), or rely on an existing
          VM runtime (which may not consider the intricacies of a functional language when executing
          the program).
\end{itemize}

The LLVM IR also provides a number of optimisations that can be applied to the program before it is
compiled to machine code, such as dead code elimination, loop unrolling, and inlining. These
optimisations can be applied to the program without the need for the language designer to implement
them, and can be tailored to the target architecture.

\section{Aims}

This project aims to implement a compiler for a small functional language, targeting the LLVM IR.
This language (inspired by the language presented in the KCL module 6CCS3CFL) will support a number
of features, including higher order functions and pattern matching. The project seeks to evaluate
the effectiveness of the LLVM toolchain as a backend for a functional language compiler, and to
explore the challenges of compiling functional languages to imperative machine code.

\section{Report Structure}

Contemporary compiler design will be explored within Chapter~\ref{ch:background}, with a focus on
translation between functional languages and the LLVM IR. As functional languages rely on the
compiler to perform more work than imperative languages, analysing how modern compilers are designed
to facilitate this is of significance. The full scope of the compiler and language will be defined
in Chapter~\ref{ch:requirements}, with the design of the compiler architecture presented in
Chapter~\ref{ch:design}. The implementation of the compiler, including how specific language
features are translated to the LLVM IR, will be discussed in Chapter~\ref{ch:implementation}.
Evaluation of the compiler, including performance in comparison to other modern compilers, will be
presented in Chapter~\ref{ch:evaluation}. An analysis of legal, social, ethical and professional
concerns related to this project is discussed in Chapter~\ref{ch:professional-issues}, with
conclusions drawn in Chapter~\ref{ch:conclusion}.
