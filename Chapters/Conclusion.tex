\chapter{Conclusion}
\label{ch:conclusion}

Targeting the LLVM as a backend for a functional language compiler has proven to be an effective
method for generating efficient machine code. The LLVM IR provides a low-level representation of
programs that can be optimised and compiled to a variety of architectures, without the need for the
compiler to target each architecture individually. Despite the complex transformations required to
compile functional langauge paradigms into imperative machine code, the SSA form of LLVM IR provides
a suitable target for this transformation.

Performance of the generated code was on par with that of the Clang compiler, with the LLVM
toolchain providing a number of optimisations that could be applied to the generated IR, such as
dead code elimination, loop unrolling, and inlining. These optimisations came at no additional cost
to the complexity of the compiler, with the LLVM toolchain providing these features out of the box.
Any future improvements to the LLVM optimiser would be inherited by the compiler, providing a
significant advantage over other compilation methods. Key features of functional languages, such as
recursion, can be efficiently compiled to LLVM IR, with the LLVM backend providing the optimisations
required to ensure the usage of such features are viable in practice.


