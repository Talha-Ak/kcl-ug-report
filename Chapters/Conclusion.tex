\chapter{Conclusion}
\label{ch:conclusion}

Targeting the LLVM as a backend for a functional language compiler has proven to be an effective
method for generating efficient machine code. The LLVM IR provides a low-level representation of
programs that can be compiled to a variety of architectures, without the need for the compiler to
target each architecture individually. Despite the complex transformations required to compile
functional language paradigms into imperative machine code, the SSA form of LLVM IR provides a
suitable target for this transformation.

Performance of the generated code was on par with that of the Clang compiler, with the LLVM
toolchain providing a number of optimisations that could be applied to the generated IR, such as
dead code elimination, loop unrolling, and inlining. These optimisations came at no additional cost
to the complexity of the compiler, with the LLVM toolchain providing these features out of the box.
Any future improvements to the LLVM optimiser would be inherited by the compiler, providing a
significant advantage over other compilation methods. Key features of functional languages, such as
recursion, can be efficiently compiled to LLVM IR, with the LLVM backend providing the tail call
optimisation necessary to prevent stack overflows.

The LLVM IR provides a portable target for functional language compilers, with the ability to target
a variety of architectures and operating systems. The LLVM project is actively maintained and
developed, with a large community of contributors and users. This ensures that the LLVM IR will
remain a viable target for compilers in the future, with new features and optimisations being added
regularly.

Areas of improvement for this compiler lie in an improved type system, with the ability to handle
more complex types such as algebraic data types and type classes. As discussed in
Chapter~\ref{ch:evaluation}, the current type system is limited in its expressiveness. Improvements
to the type system would allow for more complex programs to be compiled, with the ability to handle
more advanced features of functional languages.
