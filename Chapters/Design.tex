\chapter{Specification \& Design}
\label{ch:design}

Mirroring the convention for modern compilers, the compiler will be implemented with distinct phases
for each major transformation step outlined in the requirements. This clear separation of concerns
allows for each phase to be developed and tested independently, and for the compiler to be easily
extended or modified in the future. The phases are as follows:

\begin{enumerate}
      \item \textbf{Parsing}: Responsible for transforming the source code into an Abstract Syntax
            Tree (AST).
      \item \textbf{IR Generation}: Transforms the AST into an intermediate representation (IR)
            based on Administrative Normal Form (ANF).
      \item \textbf{Closure Conversion}: Converts the IR into a form that supports higher-order
            functions by introducing closures.
      \item \textbf{Hoisting}: Hoists nested functions and other declarations to the top-level
            global scope.
      \item \textbf{LLVM Generation}: Generates the LLVM IR from the ANF IR, and any additional
            runtime functions required.
\end{enumerate}

This is illustrated in
Figure~\ref{fig:project-compiler-spec}.

\begin{figure}
      \centering
      \includestandalone[width=\textwidth]{Graphics/project-compiler-spec}
      \caption{A diagram of each phase of the proposed compiler.}
      \label{fig:project-compiler-spec}
\end{figure}

\section{Technologies}

The language selected for this compiler is \emph{Scala}, a JVM-based functional programming
language. This choice of language was primarily motivated by the concise and expressive syntax Scala
provides, such as operator overloading and extensive support for pattern matching. However, the
choice of language is not a major factor in dictating the design of the compiler, as many of the
implementation and design decisions could easily be ported to other languages.

The \emph{ScalaTest} library will be used for testing, ensuring that the compiler is semantically
correct, as well as for identifying any regressions in the compiler's behaviour.

The LLVM static compiler \texttt{llc} will be used to generate the object file required for linking
the final executable with \texttt{gcc}. Using \texttt{llc} provides access to the same compiler
optimisations as seen in \texttt{clang} (i.e., \texttt{-O1, -02, -03}) and other compilers that
utilise the LLVM optimiser, negating the need for implementing certain optimisations within the
compiler itself.

\section{Compiler Phases}

\subsection{Parsing}

This phase is responsible for transforming the stream of characters from a source file into an AST.
The implementation will be carried out using \emph{FastParse}, a parser combinator library for
Scala. This decision was motivated by the library's parsing performance in comparison to other
parser libraries, extensive tooling for debugging and testing, in addition to its judicious use of
Scala's language features. This results in a syntax for defining parsers that is modular and simple
to reason about. An advantage of using a parser combinator library is that it allows for the
implementation to closely resemble the grammar of the language being parsed, making it easier to
maintain and extend the parser.

The trade-off when using \emph{FastParse} is that it operates directly with the stream of
characters, rather than a stream of tokens (known as \emph{scannerless parsing}). Therefore, the
lexical analysis phase will be omitted, with the parser responsible for identifying tokens.

The context-free grammar used by the parser must be unambiguous, and
(particularly for arithmetic and boolean expressions) correctly reflect the precedence and
associativity of operators. While technically possible for parser combinators to handle
ambiguous grammars \autocite{frost2008recursiveparsers}, the \emph{FastParse} library does not.

\subsection{IR Generation}

As the LLVM IR is specified to be in SSA form, compilers targeting the LLVM IR must convert their
programs into SSA form before generating LLVM IR. For functional languages, it is useful to convert
to another intermediate representation that is close to SSA form before converting to LLVM IR.

Two such intermediate representations commonly used for compiling functional languages are:
\begin{itemize}
    \onehalfspacing
    \item \textbf{Continuation-Passing Style}
    \item \textbf{A-Normal Form}
\end{itemize}

These specific representations were chosen as they both have been demonstrated to map closely to
lambda calculus \autocite{morrisett1999systemf,flanagan1993essence}, making them ideal for
representing functional languages.

\subsubsection{Continuation-Passing Style}
\label{sec:cps}

Continuation-Passing Style (CPS) refers to a programming style where, instead of function
invocations returning a value, functions instead invoke a continuation function with the value as
an argument.

For instance, the following Scala expression:

\begin{code}{scala}
    foo(1, 2) + bar(3, 4)
\end{code}

\noindent could be converted into the following CPS:

\begin{code}{scala}
    // foo and bar accept and return additional continuation function arg
    foo(1, 2,
        (a) => bar(3, 4,
            (b) => a + b
        )
    )
\end{code}

\noindent where \texttt{k} represents the continuation.

While not identical to SSA, it has been proven that a subset of CPS is equivalent to SSA, and this
subset is sufficient for use as an intermediate representation \autocite{kelsey1993correspondence}.

\subsubsection{A-Normal Form}

A simpler, more compact IR is Administrative Normal Form (ANF), referring to a programming style
where all expressions are either variables or function calls with atomic arguments.

For instance, the same Scala code from Section~\ref{sec:cps} could be converted into the following
ANF code:

\begin{code}{Scala}
    val a = foo(1, 2)
    val b = bar(3, 4)
    a + b
\end{code}

Both of these IR forms have historically been implemented in the Glasgow Haskell Compiler's (GHC).
The initial implementation of an intermediate representation made use of CPS
conversion~\autocite{ghccps}, but later switched to a direct, functional IR, akin to
ANF~\autocite{ghcanf}. The motivation for migrating IR forms was in part due the ease of reasoning
about programs and compiler optimisations~\autocite{maurer2017compiling}.

For the project, the resulting AST from the parser will be transformed into an intermediate
representation (IR), based on ANF. An ANF IR was selected over a CPS IR due to its ease in
converting to the eventual LLVM IR.

Within this phase, enumerated types are also de-sugared into integer constants, and pattern matching
is de-sugared into a series of \texttt{if} statements.

\subsection{Closure Conversion}

Supporting higher-order functions requires the introduction of closures, due to the need to capture
free variables (variables that are not defined within the function) present inside a function. This
phase is done in multiple steps:

\begin{enumerate}
      \item \textbf{Identify free variables}: Free variables are identified by analysing variables
            used but not declared within a function.

      \item \textbf{Create closure}: A closure is created for each function that contains free
            variables. This closure is a structure that contains a pointer to the function, and the
            values of the free variables.

      \item \textbf{Rewrite body}: The body of the function is rewritten to reference the free
            variables from the closure, rather than the original free variables.

      \item \textbf{Rewrite calls}: Calls to the function are rewritten to pass the closure as an
            additional parameter. The function pointer is extracted from the closure, and the
            function is called with the environment as an additional parameter.
\end{enumerate}

\subsection{Hoisting}

Hoisting is the process of moving nested functions and other declarations to the top-level global
scope. This ensures that all declarations are visible to the entire program, and that functions can
be called from anywhere in the program. This is necessary for the LLVM IR, as functions must be
declared before they are called.

\subsection{LLVM Generation}

This final phase is responsible for traversing the ANF IR and generating the corresponding LLVM IR.
Any additional runtime functions required for supporting the language features are also generated
within this phase, such as functions to \texttt{print()}. It is imperative that the type information
from the source language is preserved up to this point, as the LLVM IR is strongly typed. The
generated LLVM IR is then written to a file, ready for compilation with \texttt{llc}.
