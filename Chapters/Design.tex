\chapter{Specification \& Design}

Mirroring the convention for modern compilers, the compiler will be implemented with distinct phases
for each major transformation step outlined in the requirements. This is illustrated in
Figure~\ref{fig:project-compiler-spec}.

\begin{figure}
    \centering
    \includestandalone[width=\textwidth]{Graphics/project-compiler-spec}
    \caption{A diagram of each phase of the proposed compiler.}
    \label{fig:project-compiler-spec}
\end{figure}

\section{Technologies}

The language selected for this compiler is \emph{Scala}, a JVM-based functional programming
language. This choice of language was primarily motivated by the concise and expressive syntax Scala
provides, such as operator overloading and extensive support for pattern matching. However, the
choice of language is not a major factor in dictating the design of the compiler, as many of the
implementation and design decisions could easily be ported to other languages.

% ... will invoke llvm backend to generate executable
% ... using mill build system
% ... test library

\section{Compiler Phases}

\subsection{Parser}

This phase is responsible for transforming the stream of characters from a source file into an AST.
The implementation will be carried out using \emph{FastParse}, a parser combinator library for
Scala. This decision was motivated by the library's parsing performance in comparison to other
parser libraries, extensive tooling for debugging and testing, in addition to its judicious use of
Scala's language features. This results in a syntax for defining parsers that is modular and simple
to reason about.

\subsection{IRGen}

\subsection{ClosureConv}

\subsection{LLVMGen}
