\chapter{Legal, Social, Ethical and Professional Issues}
\label{ch:professional-issues}

This chapter discusses the implications of the British Computing Society (BCS) Code of Conduct on
the project, as well as other legal, social, ethical and professional considerations that may affect
the project. While a project on a compiler may not necessarily have as many legal and ethical
considerations in comparison to other projects, it is still important to consider these issues in
order to ensure that the project is conducted professionally.

\section{The BCS Code of Conduct}

The BCS define a code of conduct that `set rules and professional standards to direct the behaviour
of its members in professional matters.'~\autocite{bcs}. The code of conduct highlights four key
principles that must be adhered to:

\begin{enumerate}
    \item \textbf{Public Interest:} Due regard must be given to the safety and interests of others.
          The relevance of this principle to the development of a compiler is limited, as the effect
          of a compiler on the public is minimal.

          Theoretically, if the compiler were to be distributed as a binary, a malicious actor could
          inject malicious code into the compiler, infecting all programs compiled with it. A
          similar vulnerability is a compiler backdoor, where bootstrapped compilers (compilers
          compiled with themselves) could contain a malicious payload that replicates itself in all
          future versions of the compiler.~\autocite{thompson1984reflection}

          However, the source code of this project is open-source, and not bootstrapped (i.e. it is
          not compiled with itself, but with Scala), and is therefore not a realistic concern.

    \item \textbf{Professional Competence and Integrity:} The work carried out must be done to the
          best of the one's ability, without misrepresenting one's skills or knowledge. Within the
          remit of this project, justification was provided for the decisions that were made, and
          where possible were built upon reputable sources.

    \item \textbf{Duty to Relevant Authority:} Care must be taken to ensure that work completed
          under the Relevant Authority (in this case, King's College London) meets their
          requirements, and that any conflicts of interest are declared. Throughout the project, the
          Academic Regulations of the College, the requirements of module 6CCS3PRJ, as well as the
          Academic Honesty and Integrity Policy, were adhered to. No conflicts of interest were
          present.

    \item \textbf{Duty to the Profession:} The reputation of the profession must be upheld as a
          personal duty, acting in a way that promotes the profession. This project is presented as
          a case study for the implementation of functional language constructs in LLVM IR, where
          others are able to learn and improve upon the work done here.
\end{enumerate}

\section{Other implications}

The software licence chosen for a compiler is important, as it determines how the software can be
used, modified and distributed. Not only does the licence affect the software itself, but it may
also potentially effect the software that is compiled with it.

For instance, if the runtime library of a compiler is licensed under the `GNU General Public
License' (GPL), then any software compiled with the compiler must also be licensed under the GPL.
This is because the library is linked with the compiled software, and the GPL requires that any
software linked with GPL software must also be licensed under the GPL.

In fact, the GCC compiler \emph{does} licence its runtime libraries under the GPL, which is why the
authors of GCC include a GCC Runtime Library Exception, allowing the compilation of non-GPL software
with certain GPL-licensed library files~\autocite{gcc-exception}. Similarly, the LLVM project is
licenced under the Apache 2.0 licence `with LLVM Exceptions'. This allows for certain licence conditions
to be ignored when redistributing embedded LLVM code in a compiled binary~\autocite{llvm-licence}.

An appropriate licence for this project is the MIT Licence, which allows for the software to be
used, modified and distributed freely, as long as the original licence and copyright notice is
included. This licence is permissive, and allows for the software to be used in both open-source and
proprietary projects.
