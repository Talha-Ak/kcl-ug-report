\chapter{Language Specification}
\label{app:grammar}

The syntax is based on the syntax of the Scala language, with major
simplifications (e.g.\ the lack of classes, objects, and inheritance) for the purposes of focusing
on the compilation of functional language features.

The following specifies the syntax of the language created in this project, presented in Extended
Backus-Naur Form (EBNF). For clarity, the following conventions for the EBNF grammar will be used:
\begin{itemize}
    \singlespacing
    \item \synt{...} denotes a non-terminal symbol.
    \item \lit{...} denotes a terminal symbol.
    \item \syntax{(...)} denotes a grouping of symbols.
    \item \syntax{\synt{...} | \synt{...}} denotes a choice between two or more alternatives.
    \item \syntax{\{ \synt{...} \}} denotes zero or more repetitions of the enclosed
          expression.
    \item \syntax{[ \synt{...} ]} denotes an optional (zero or one repetition)
          expression.
\end{itemize}

\section{Lexical Syntax}

\begin{grammar}

    <identifier> ::= <letter> \{ <letter> | <digit> | `_' \}

    <letter> ::= `A' | ... | `z'

    <digit> ::= `0' | ... | `9'

    <number> ::= <digit> \{ <digit> \}

    <float> ::= <number> `.' <number>

    <bool> ::= `true' | `false'


\end{grammar}

\section{Keywords}

The following names are reserved words, rather than identifiers.

\begin{multicols}{4}
    \singlespacing
    \begin{itemize}
        \item \lit{def}
        \item \lit{val}
        \item \lit{enum}
        \item \lit{struct}
        \item \lit{if}
        \item \lit{then}
        \item \lit{else}
        \item \lit{write}
        \item \lit{true}
        \item \lit{false}
    \end{itemize}
\end{multicols}

\section{Grammar}

\begin{grammar}

    <program> ::= `def main () =' <expr>
    \alt <def fn> <program>
    \alt <defenum> <program>
    \alt <defstruct> <program>

    <def fn> ::= `def' <identifier> `(' \{ <typed ident> `,' \} `) :' <type>  `=' <expr> `;'

    <def enum> ::= `enum' <identifier>  `=' <identifier> \{ `|' <identifier> \} `;'

    <def struct> ::= `struct' <identifier> `= {' <typed ident> \{ `,' <typed ident> \} `};'

    <def val> ::= `val' <typed ident> `=' <expr> `;'

    <typed ident> ::= <identifier> `:' <type>

    <type> ::= `int'
    \alt `bool'
    \alt `void'
    \alt `float'
    \alt `(' \{ <type> `,' \} `) =>' <type>
    \alt <identifier>

    <expr> ::= `if' <equal> `then' <expr> `else' <expr>
    \alt `print (' <equal> `)'
    \alt <match>
    \alt <block>
    \alt <equal>

    <match> ::= <identifier> `match {' <case> \{ <case> \} `}'

    <case> ::= `case' <identifier>`::'<identifier> `=>' <expr>
    \alt `case _ =>' <expr>

    <block> ::= `{' \{ <def fn> | <def val> \} <expr> \{ `;' <expr> \} `}'

    <equal> ::= <comp> \{ (`==' | `!=') <comp> \}

    <comp> : <term> \{ (`<' | `>' | `>=' | `<=') <term> \}

    <term> ::= <fact> \{ (`+' | `-') <fact> \}

    <fact> ::= <primary> \{ (`*' | `/' | `\%') <primary> \}

    <primary> ::= <identifier> `(' \{ <expr> `,' \} `)'
    \alt `(' <expr> `)'
    \alt <identifier>`::'<identifier>
    \alt <identifier>`.'<identifier>
    \alt <number>
    \alt <float>
    \alt <bool>
    \alt <identifier>

\end{grammar}
