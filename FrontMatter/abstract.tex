Functional programming languages are characterised by their focus on functions as first-class
objects, with programs being constructed by composing functions together. Typical targets for
compilers, such as the LLVM IR, are designed with imperative languages in mind, and as such, require
frontends to convert from a functional paradigm to an imperative one.

This project aims to explore the effectiveness of implementing functional language features that
target the LLVM IR, and how well the LLVM toolchain can optimise and compile these features to
machine code.

Results show that, after applying the LLVM optimiser to the generated IR, the performance of the
compiled code is comparable to that of other modern compilers such as Clang. The LLVM toolchain
provides a powerful backend for functional language compilers, with the ability to optimise the
generated IR for a variety of target architectures.
